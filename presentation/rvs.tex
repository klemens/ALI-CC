\usepackage[ngerman]{babel}
\usepackage{graphicx}
\usepackage{stmaryrd}
\usepackage{listings}
\usepackage{amsmath}
\usepackage{hyperref}
\usepackage{tikz}

% theme settings
\usetheme{default}
\usefonttheme{structurebold}
\usecolortheme{default}
\definecolor{rvs-darkred}{RGB}{192,0,0}
\definecolor{rvs-darkgreen}{RGB}{0,130,0}
\definecolor{rvs-darkgrey}{RGB}{70,70,70}
\setbeamercolor*{title}{fg=rvs-darkred}
\setbeamercolor*{frametitle}{fg=rvs-darkred}
\setbeamercolor*{block title}{fg=rvs-darkgrey}
\setbeamercolor*{block title example}{fg=rvs-darkgreen}
\setbeamercolor*{section in toc}{fg=rvs-darkgrey}
\setbeamercolor*{item}{fg=black}
\setbeamertemplate{navigation symbols}{}
\setbeamertemplate{itemize items}[circle]
\setbeamertemplate{footline}{
    \usebeamercolor[fg]{page number in head/foot}
    \usebeamerfont{page number in head/foot}
    \hspace{.5em}
    Anwendungen Linguistische Informatik
    \hfill
    \insertframenumber\kern.5em\vskip.5em
}
\defbeamertemplate*{subsection in toc}{sub on 1 line}{
    \ifnum\inserttocsubsectionnumber=1
        \quad\inserttocsubsection
    \else
        \hspace{-1em},\hspace{.5em}\inserttocsubsection
    \fi
}

% tikz graph settings
\usetikzlibrary{arrows,positioning,calc}
\tikzstyle{vertex}=[inner sep=10pt]
\tikzstyle{edge}=[line width=1]

\theoremstyle{definition}
\newtheorem*{algorithm}{Algorithmus}
\newtheorem*{observation}{Beobachtung}
